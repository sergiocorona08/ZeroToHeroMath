\documentclass{article} % Tipo de documento
\usepackage{amsthm}
\usepackage[utf8]{inputenc} % Codificación
\usepackage[spanish]{babel} % Idioma
\usepackage{amsmath, amssymb}
\usepackage[a4paper, margin=1.5cm]{geometry} % Márgenes de 1.5 cm

\title{Guía de formatos para LaTeX y ejemplos de formulas}
\author{Sergio Augusto Macías Corona}
\date{\today} % Fecha actual

\begin{document}

\maketitle

\section{Modo Matemático}
En línea: para expresiones dentro del texto:
La ecuación es $a^{10} + \frac{1}{2}$\\
\\
Ecuaciones centradas (display mode):
\[
a^2 + b^2 = c^2
\]

Para numerar las ecuaciones del documento
\begin{equation}
a^2 + b^2 = c^2
\end{equation}

\section{Simbolos básicos}

Fracción: $\frac{a}{b}$\\
Potencia: $x^{2}$\\
Subindice: $x_{i}$\\
Raíz cuadrada: $\sqrt{x}$\\
Raíz n-ésima: $\sqrt[n]{x}$\\
Sumatoria: $\sum_{i=1}^{n} a_i$\\
Producto: $\prod_{i=1}^{n} a_i$\\
Notación prime: $f'(x)$\\
Derivada: $\frac{dy}{dx}$\\
Derivada parcial: $\frac{\partial f}{\partial x}$\\
Integral: $\int_{a}^{b} f(x) dx$\\
Límite: $\lim_{x \to 0} f(x)$\\
Aproximado: $\approx$\\
Pertenece: $\in$\\
Conjunto vacío: $\emptyset$\\
Infinito: $\infty$\\

\section{Matrices y Sistemas de ecuaciones}

\[
\begin{bmatrix}
1 & 2 \\
3 & 4
\end{bmatrix}
\]

\[
\begin{cases}
x + y = 1 \\
2x - y = 3
\end{cases}
\]

\section{Letras Griegas}
\begin{itemize}
  \item $\alpha$
  \item $\beta$
  \item $\gamma$
  \item $\delta$
  \item $\epsilon$
  \item $\lambda$
  \item $\pi$
  \item $\sigma$
  \item $\phi$
  \item $\omega$
\end{itemize}


\section{Relaciones y símbolos lógicos}

Pertenencia: $\in$ \\
No pertenece: $\notin$ \\
Subconjunto: $\subset$ \\
Subconjunto propio: $\subsetneq$ \\
Inclusión: $\subseteq$ \\
No inclusión: $\nsubseteq$ \\
Existencia: $\exists$ \\
Para todo: $\forall$ \\
Negación: $\neg$ \\
Y lógico (conjunción): $\land$ \\
O lógico (disyunción): $\lor$ \\
Implicación: $\implies$ \\
Equivalencia: $\iff$ \\
Divisibilidad: $a \mid b$ \\
No divisibilidad: $a \nmid b$ \\
Factorial: $n!$

\section{Texto dentro de formulas}
\[
x = 0 \quad \text{si y sólo si} \quad y = 0
\]

\section{Funciones comunes}

Seno: $\sin(x)$ \\
Coseno: $\cos(x)$ \\
Tangente: $\tan(x)$ \\
Cotangente: $\cot(x)$ \\
Secante: $\sec(x)$ \\
Cosecante: $\csc(x)$ \\
Logaritmo natural: $\ln(x)$ \\
Logaritmo base $b$: $\log_b(x)$ \\
Valor absoluto: $|x|$ \\
Parte entera: $\lfloor x \rfloor$ \\
Parte techo: $\lceil x \rceil$

\section{Atajos utiles}
\begin{align*}
a + b &= c \\
d + e &= f
\end{align*}
\textbf{Texto en negritas}

\section{Ambientes de Teorema, Definición y Ejemplo}

\newtheorem{teorema}{Teorema}
\newtheorem{definicion}{Definición}
\newtheorem{ejemplo}{Ejemplo}

\begin{definicion}
Un número primo es aquel que sólo tiene dos divisores: $1$ y él mismo.
\end{definicion}

\begin{teorema}
Si $p$ es un número primo y $p \mid ab$, entonces $p \mid a$ o $p \mid b$.
\end{teorema}

\begin{ejemplo}
El número $7$ es primo ya que sólo es divisible entre $1$ y $7$.
\end{ejemplo}



\end{document}