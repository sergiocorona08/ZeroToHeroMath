\documentclass{article} % Tipo de documento
\usepackage[a4paper, margin=1.5cm]{geometry} % Márgenes de 1.5 cm
\usepackage[utf8]{inputenc} % Codificación
\usepackage[T1]{fontenc} %Nos sirve para agregar texto
\usepackage{amsmath, amssymb}
\usepackage{graphicx}
\usepackage{float}
\usepackage{forest}

\title{Zero to Hero, Math}
\author{Sergio Augusto Macías Corona}
\date{\today} % Fecha actual

\begin{document}

\maketitle

\section{Introduction}

The following PDF will be a very useful tool created by me to help me along my professional carreer. It will also help me review anything I might have forgotten over time abouth Math, and why not? Also help me brush up on my English writing. If you are reading this, maybe it´s just me, but if not. Just want to tell you that I´m shit at Math but I hope that with hard work, sweat, and many tears, me, but also you can learn everything you don't understand about math and be able to complete your college, high school, or whatever else is in a more enjoyable way, and not suffer as much as I did. With love, sergio\_corona88 at Instagram.

\section{Level 0: Absolute Fundamentals (Pre - Math)}
\subsection{Natural numbers}
Natural numbers are the set of all the numbers from one to infinity excluding zero. (1 - $\infty$).The set of natural numbers is denoted by the symbol, N. So, the set of natural numbers is represented as $N = [1,2,3,4,5 \dots \infty$]. 

\subsection{Addition and subtraction}
The two simplest operations: addition is used to find the total between two numbers. $a = b + c$ or $4 + 2 = 6$ Subtraction, on the other hand, is used to find the difference between two numbers. $a = b - c$ or $6 - 4 = 2$

\subsection{Multiplication and division}
Multiplication and division are directly related because they are essentially inverse operations. Multiplication seeks to join equal groups, while division seeks to separate these equal groups.

Let's start with the most basic. If we have $4 \times 5 = 20$, their inverse relationships (in the form of division) would be the following:

\[
20 \div 5 = 4
\]

\[
20 \div 4 = 5
\]

De igual forma, si tomamos la división $30 \div 3 = 10$, sus relaciones inversas (en forma de multiplicación) serían las siguientes:
\[
3 \times 10 = 30
\]

\[
10 \times 3 = 30
\]
In both examples, we can see that we use the same three numbers. This is because when we multiply two numbers (which we call factors), we obtain a result that we call the product. If we divide that number by one of the factors, we obtain the other factor as the result.

\subsection{Even and odd numbers}
Basically, an even number is one that can be divisible by 2 and generates a remainder of 0. An even number can be easily divided into equal groups. $(2,4,6,8,10\dots)$.
In the other hand, odd numbers are numbers that are not evenly divisible by 2. When divided by 2, even numbers leave a remainder of 1. $(1,3,4,5,7,9\dots)$.

\subsection{Number comparison}
Comparing numbers in the process in which we can determine whether a number is smaller, greater, or equal to another number according to their values. The symbols used for comparing numbers are “>”, which means “greater than”; “<”, which means “less than”; and “=”, which means “equal to". We also have a subcategory type that allows us to determine whether a number is, for example, greater than another but also equal. This is defined by the following symbol "$\geq$" which means "greater than or equal to". And finally, "less than or equal to," which is defined by the following symbol "$\leq$".

\[
6 > 3
\]
6 is greater than 3 wich is true. 

\[
2 > 9
\]
2 is smaller than 9 wich is true.

\[
5 = 5
\]
5 is equal to 5 wich is also true.

\subsection{Zero and negative numbers}
Positive numbers represent an amount of something. Negative numbers represent taking away an amount of something. $(-1,-2,-3,-4,-5\dots)$ For example if we acquire some amount of money we can represent it by a positive number. If we then spend some of that money we can represent it by a negative number.

Zero as a number represents a starting point when we have neither acquired something nor had any of it taken away. Is a digit that is crucial to place value number systems. It allows us to represent numbers that have none of a particular place value.

\section{Level 1: Basic arithmetic}

\subsection{Whole numbers}
The whole numbers are the part of the number system which includes all the positive integers from 0 to infinity. These numbers exist in the number line. Hence, they are all real numbers. $(0, 1, 2, 3, 4 \dots \infty)$

\subsection{Fractional numbers}
The fractional numbers are numbers that represent one or more parts of a unit that has been divided in equal parts. They are figured out by two whole numbers (the fraction terms) that are separated by a horizontal line (the fraction line). The number above the line (the numerator) can be every whole number and the number below the line (the denominator) should be different from zero.

There can be different types of fractional numbers which are named differently as follows.\\

\textbf{Proper Fraction:} the number is inferior to the denominator, for instance:
\[
\frac{3}{4}
\]

\textbf{Improper Fraction:} the numerator is superior to the denominator, for instance:
\[
\frac{9}{2}
\]

\textbf{Mixed Fraction or Mixed Numeral:} it is composed of a whole part and a fractional one, for instance:
\[
2\frac{1}{3}
\]

\textbf{Equivalent Fractions:} fractions that keep on the same proportion of another fraction, for instance:
\[
\frac{5}{2} = \frac{10}{4}
\]

\textbf{Irreducible Fraction:} it cannot be simplified, for instance:
\[
\frac{4}{3}
\]

\textbf{Decimal Fraction:} the denominator is a power whose base is $10(10, 100, 1000 \dots)$, for instance:
\[
\frac{8}{10}
\]
Not every number written as a fraction is a fractional number,the reason is that fractional numbers represent one or more parts of a whole. For example, the fraction $\frac{10}{2}$ is written as a fraction, but it is not a fractional number because it simplifies to $5$, which is a whole number and does not represent a part of a whole. Similarly, $\frac{\sqrt{2}}{3}$ is expressed as a fraction, but it is not a fractional number because the numerator ($\sqrt{2}$) is not an whole number.

\subsection{Decimal numbers}
Decimals are one of the types of numbers, which has a whole number and the fractional part separated by a decimal point. The dot present between the whole number and fractions part is called the decimal point. For example, $34.5$ is a decimal number.

There can be different types of decimal numbers which are named differently as follows.\\

\textbf{Recurring Decimal Numbers:} (Repeating or Non-Terminating Decimals)

\[
3.125125 (Finite)
\]
\[
3.121212121212\dots (Infinite)
\]

\textbf{Non-Recurring Decimal Numbers:} (Non Repeating or Terminating Decimals)

\[
3.2376 (Finite)
\]
\[
3.137654\dots (Infinite)
\]

\textbf{Decimal Fraction:} (It represents the fraction whose denominator in powers of ten.)

\[
81.75 = \frac{81750}{1000} 
\]

\subsection{Percentages}
A percentage is a number, or ratio, that expresses a fraction out of 100. Percentages are easy to recognize because they are always followed by a percent sign (\%). A percentage is an alternative way to represent a fraction out of 100 instead of using a traditional fraction format. For example, we can write $\frac{1}{2}$ or $\frac{50}{100}$ to represent a half fraction, using percentage we can also write 50\%.

\subsection{Proportions and ratios}
A ratio compares two or more quantities, showing their relative sizes. It can be written in different ways:

\textbf{Colon notation:} $a : b$ (example, $2:3$)

\textbf{Fraction form:} $a / b$ (example, $2 / 3$)

\textbf{'to' phrasing:} $a / b$ (example, 2 to 3)\\

A proportion is a statement that two ratios (or fractions) are equal. It tells us that two different comparisons have the same relationship.
If $\frac{a}{b}$ and $\frac{c}{d}$ represent the same relative amount, then:

\[
\frac{a}{b} = \frac{c}{d}
\]

\subsection{Least common multiple (LCM) and greatest common divisor (GCD)}
The GCD of two or more integers is the largest positive integer that divides each of the given numbers without leaving a remainder. In other words, it is the greatest common factor shared by the numbers.

\textbf{For example to:}\\
\noindent The GCD of:
\[
\{34, 56\} \text{ is } 22
\]
\noindent The GCD of:
\[
\{64, 96\} \text{ is } 32
\]

The LCM of two or more integers is the smallest positive integer that is divisible by each of the given numbers. It is the least common multiple shared by the numbers.

\textbf{For example to:}\\
\noindent The LCM of:
\[
\{34, 56\} \text{ is } 952
\]
\noindent The LCM of:
\[
\{64, 96\} \text{ is } 192
\]


\subsection{Order of operations (PEMDAS)}
The order of operations is a set of rules that tells us the correct sequence to solve math problems with multiple operations (like addition, subtraction, multiplication, division, exponents, etc.). Without it, the same problem could have different and wrong answers.
So here I show a quick guide of what the order of operations is:

\textbf{Solve parentheses first:}\\
\[
4 \times (5 + 3) = 4 \times (8) = 32
\]

\textbf{Exponents (Powers, Roots) before Multiply, Divide, Add or Subtract:}\\
\[
5 \times 2^{2} = 5 \times 4 = 20
\]

\textbf{Multiply or Divide before you Add or Subtract:}\\
\[
2 + 5 \times 3 = 2 \times 15 = 17 
\]

\textbf{Otherwise just go left to right:}\\
\[
30 \div 5 = 6 \times 3 = 18
\]

\subsection{Rounding and estimating}
\textbf{Rounding numbers}\\
We do not always need to give exact answers to problems. Rounding means making a number simpler (less precise) while keeping it close to the original value. We often round to a specific place value.

\textbf{Rules for Rounding:}\\
\textbf{1.-} Identify the place value you're rounding to (tens, hundreds, decimals).\\
\textbf{2.-} Look at the digit to the right of that place:\\
If it's 5 or higher, round up the target digit by 1.\\
If it's 4 or lower, Keep the target digit the same.
\textbf{3.-} Replace all digits to the right with zeros (for whole numbers) or drop them (for decimals).\\
\\
\textbf{Estimating}\\
Estimating involves using rounded numbers to quickly approximate calculations. It´s useful for checking if an exact answer is reasonable.

\textbf{Estimating methods:}\\
\textbf{Front-End Estimation:} Use the first digits and adjust.\\
\[
327 + 589 \approx 300 + 500 = 800
\]

\noindent \textbf{Rounding First:} Round all numbers before calculating.\\
\[
23 \times 48 \approx 20 \times 50 = 1000
\]

\section{Level 2: Advanced Arithmetic / Pre - Algebra}

\subsection{Exponents and Square Roots}
An exponent is the number which says how many times to multiply the base. A square root is finding the number that is multiplied by itself to get the number under the radical. The square root is opposite of a number being squared, meaning with the exponent of two.\\
\\
There are 8 important laws of exponents to know that will make it easier for us to solve and analyze them.
\\
\textbf{Zero Rule:}\\
Any number to the power of 0 equals 1. $a^{0} = 1$\\
Example:\\
\[
5^{0} = 1
\]

\noindent\textbf{One Rule:}\\
If the exponent is 1, the result is the base itself $a^{1} = a$\\
Example:\\
\[
3^{1} = 3
\]

\noindent\textbf{Negative Exponent Rule:}\\
A negative exponent flips the number into a fraction $a^{-n} = \frac{1}{a^{n}}$\\
Example:\\
\[
2^{-3} = \frac{1}{2^{3}} = \frac{1}{8} 
\]

\noindent\textbf{Product Rule:}\\
When multiplying the same bases, add the exponents $a^{m} \times a^{n} = a^{m + n}$\\
Example:\\
\[
x^{2} \times x^{4} = x^{2 + 4} = x^{6}
\]

\noindent\textbf{Quotient Rule:}\\
When dividing the same bases, subtract the exponents $\frac{a^{m}}{a^{n}}$\\
Example:\\
\[
x^{2} \times x^{4} = x^{2 + 4} = x^{6}
\]

\noindent\textbf{Power of a Power Rule:}\\
Multiply the exponents $(a^{m})^{n}$\\
Example:\\
\[
(a^{3})^{2} = a^3\times 2 = a^{6} = 64
\]

\noindent\textbf{Fractional Exponent Rule:}\\
A fractional exponent represents a root $a^{\frac{m}{n}} = \sqrt[n]{a^{m}}$\\
Example:\\
\[
8^{\frac{3}{2}} = \sqrt[3]{8^{2}} = \sqrt[3]{64} = 4
\]


\begin{figure}[H]
    \centering
    \includegraphics[width=0.5\textwidth]{Exponent.png}
    \caption{Quick overview of how the exponent works.}
    \label{fig:etiqueta}
\end{figure}

The square root of a number is a value that, when multiplied by itself, equals the original number. They can be divided into several types.
\\
\textbf{Basic Examples}\\
\\
\textbf{Exact root:}\\
Example:\\
\[
\sqrt{9} = 3 \times 3 = 9
\]
\\
\textbf{Non-exact root:}\\
Example:\\
\[
\sqrt{2} = 1.414 \text{ it is irrational, it has infinite decimals}
\]
\textbf{Key Properties}\\
\\
\textbf{For positive numbers only:}\\
$\sqrt{-4} \text{ is not a real number (but exists in complex numbers like \(2i\)).}$
\\
\textbf{Square root of zero:}\\
$\sqrt{0} = 0$
\\
\textbf{Relationship with exponents:}\\
$\sqrt{x} = x^{\frac{1}{2}}$

\begin{figure}[H]
    \centering
    \includegraphics[width=0.5\textwidth]{Root.png}
    \caption{Quick overview of how the roots works.}
    \label{fig:etiqueta}
\end{figure}

\subsection{Scientific notation}
Scientific notation is a way of writing numbers that is commonly used in science for its convenience and to reduce the likelihood of making errors. One of the typical examples is the mass of the electron, which in decimal notation is:
\[
0.000000000000000000000000000000911 \text{ Kg}
\]
Written in scientific notation it is:
\[
9.11 \times 10^{-31} \text{ Kg}
\]
A number written in scientific notation consists of two parts:\\
\textbf{The mantissa:} decimal part (on the left), whose whole number is a single digit and different from $0$.\\
\textbf{The order of magnitude:} power of base $10$ (on the right).\\
As the last example, the mantissa is $9.11$ and the order of magnitude is $10^{31}$ (base $10$ power and negative exponent $-31$):
\begin{figure}[H]
    \centering
    \includegraphics[width=0.5\textwidth]{SciNot.jpg}
    \caption{Quick overview of how the Scientific Notation works.}
    \label{fig:etiqueta}
\end{figure}

\subsection{Divisibility rules}
A divisibility rule is a heuristic for determining whether a positive integer can be evenly divided by another (with no remainder left over). For example, determining if a number is even is as simple as checking to see if its last digit is $2, 4, 6, 8$ or $0$. Multiple divisibility rules applied to the same number in this way can help quickly determine its prime factorization without having to guess at its prime factors.\\
A positive integer $N$ is divisible by:\\
$2$ If the last digit of $N$ is $2, 4, 6, 8,$ or $0$;\\
$3$ if the sum of digits of $N$ is a multiple of $3$;\\
$4$ if the last $2$ digits of $N$ are a multiple of $4$;\\
$5$ if the last digit of $N$ is either $0$ or $5$;\\
$6$ if $N$ is divisible by both $2$ and $3$;\\
$7$ if subtracting twice the last digit of $N$ from the remaining digits gives a multiple of $7$;\\
$8$ if the last $3$ digits of $N$ are a multiple of  $8$;\\
$9$ if the sum of digits of $N$ is a multiple of $9$;\\
$10$ if the last digit of $N$ is $0$;\\
$11$ if the difference of the alternating sum of digits of $N$ is a multiple of $11$;\\
$12$ if $N$ is divisible by both $3$ and $4$;\\

\subsection{Prime Factors}
Prime factors are the building blocks of numbers, they're the prime numbers that multiply together to create the original number. For example, The number $15$ can be broken down into $3 \times 5$. Since both $3$ and $5$ are prime numbers, they're the prime factors of $15$.\\
They are two methods to find Prime Factors:\\
\textbf{Factor tree}\\
Start with your number and find any two factors. Keep breaking down each factor until you're left with only prime numbers.\\
\begin{center}
\begin{forest}
for tree={circle,draw, l sep=20pt, s sep=15pt}
[24
    [4
        [2]
        [2]
    ]
    [6
        [2]
        [3]
    ]
]
\end{forest}
\end{center}
\textbf{Division by primes}\\
Divide by the smallest prime ($2$) until you can't, then, continue with the other primes ($3, 5, 7 \dots$) until you reach $1$.\\
For example:\\
\[
36 \div 2 = 18\\
18 \div 2 = 9\\
9 \div 3 = 3\\
3 \div 3 = 1\\
\]
Prime factors: $2 \times 2 \times 3 \times 3$ (or $2^{2} \times 3^{2}$).

\section{Level 3: Algebra I}
\subsection{Variables and algebraic expressions}
A variable is a symbol (usually a letter like $x, y, a, b$) that represents an unknown or changeable value in mathematics.\\
To generalize problems ("a number" $x$) or model real-world relationships (for example distance $=$ speed $\times$ time) $d =s \times t$.\\
\\
\textbf{An algebraic expression is a combination of:}\\
Variables: (For example: $x, \pi, \lambda$)\\
Numbers: (For example: $2, -5, \frac{1}{3}$)\\
Operations: (For example: $+ , - , \times , \div $)\\
Here are some examples:\\
$3x + 4$\\
$2a^{2}-5b+7$\\
\\
\textbf{Key Components of Algebraic Expressions}\\
Constant: A fixed number (no variable). Example (In a $3x + 5$, $5$ is a constant).\\
Coefficient: The number multiplying a variable. Example (In $-4y$, $-4$ is the coefficient).\\
Term: A single part of an expression (separated by $+$ or a $-$). Example: ($2x$ and $3$ are terms in $2x+3$).\\
\\
\textbf{Types of Expressions}\\
Monomial: Single term ($7x$).\\
Binomial: Two terms ($3y - 2$).\\
Polynomial: Multiple terms ($4x^{2}+x-5$)\\
\\
\textbf{Evaluating Expressions}\\
To evaluate an expression, substitute variables with given numbers and simplify.\\
For example:\\
Evaluate $2x+3$ when $x = 4$:
\[
2(4)+3 = 8 + 3 = 11
\]
\textbf{Simplifying Expressions}\\
Combine like terms (terms with the same variable and exponent):\\
For example:\\
Simplify $3x+2y-x+4$:\\
1.- Group by terms: $(3x-x)+2y+4$\\
2.- Combine them: $2x+2y+4$\\
\\
\textbf{Common Operations}\\
\textbf{Addition/Subtraction:} Only combine like terms.\\
For example:\\
$5x + 2x = 7x$ its correct\\
$3a + 4b$ (cannot simplify)\\
\\
\textbf{Multiplication:} Use the distributive property $a(b+c) = ab + ac$.\\
For example:\\
$2(x + 3) = 2x + 6$\\
\\
\textbf{Division:} Split into separate terms.\\
For example:\\
$\frac{6x + 4}{2} = 3x + 2$\\
\\
\textbf{Real-World Applications}\\

Physics: $F = ma$ (force $=$ mass $\times$ acceleration).\\

\textbf{Finance:}\\
$I = Prt$ (principal $\times$ rate $\times$ time).\\

\textbf{Geometry:}\\
$A = lw$ (length $\times$ width).
\subsection{First Degree linear equations}
First-degree equations, also known as linear equations, have the general form $ax \times b = c$, where $a, b$, and $c$ are real numbers and $a \neq 0$. In this type of equation, the variable $x$ is raised to the first power (degree one), hence the name "first degree equation".\\
In the expression $ax + b = c$, the elements $a, x, b$, and $c$ have specific roles. $a$ is the coefficient of the variable $x$, $b$ is the independent term, and $c$ is the result of the expression.\\
\textbf{Solution of a First Degree Equation}\\
The solution of a first-degree equation is the value that, when substituted for the variable, makes the equality true. Equations can be manipulated in various ways, changing the position of terms or performing equivalent operations on both sides of the expression. These transformations are useful for isolating the variable $x$ and finding its solution.\\
\\
\textbf{Solving Step-by-Step}\\
To solve:\\
\[
ax + b = 0
\]
1.- Isolate the term with $x$\\
\[
ax = -b 
\]
2.- Solve for $x$\\
\[
x = \frac{-b}{a}
\]
\subsection{Systems of linear equations (2x2)}
A system of two linear equations with two variables (typically $x$ and $y$) has the general form:
\[
\begin{cases}
a_1x + b_1y = c_1 \\
a_2x + b_2y = c_2
\end{cases}
\]
There are different methods for solving systems of 2x2 linear equations.\\
\textbf{Substitution Method}\\
Steps:\\
1. Solve one equation for one variable (e.g., $x$ in the first equation).\\

2. Substitute this expression into the other equation.\\

3. Solve for the second variable ($y$).\\

4. Find the first variable ($x$) by substituting back the value obtained for the second variable.\\

\textbf{Example of Application}\\

Consider the system of equations:
\[
\begin{cases} 
x + y = 5 \\ 
2x - y = 1 
\end{cases}
\]

1. Solve for $x$ in the first equation: $x = 5 - y$.

2. Substitute into the second equation: $2(5 - y) - y = 1 \rightarrow 10 - 3y = 1$.

3. Solve for $y$: $-3y = -9 \rightarrow y = 3$.

4. Find $x$ by substitution: $x = 5 - 3 = 2$.\\

\textbf{Solution}: The solution to the system is $x = 2$, $y = 3$.\\
\\
\textbf{Elimination/Addition and Subtraction Method}\\
Steps:\\
1. Align both equations with like terms in columns.\\

2. Multiply one or both equations by constants to make the coefficients of one variable opposites.\\

3. Add the equations to eliminate one variable.\\

4. Solve for the remaining variable.\\

5. Substitute back to find the eliminated variable.\\

\textbf{Example of Application}\\

Consider the system of equations:
\[
\begin{cases} 
2x + 3y = 8 \\ 
4x - y = 6 
\end{cases}
\]

1. Original system:
\[
\begin{cases} 
2x + 3y = 8 \\ 
4x - y = 6 
\end{cases}
\]

2. Multiply the second equation by 3 to match coefficients:
\[
\begin{cases} 
2x + 3y = 8 \\ 
12x - 3y = 18 
\end{cases}
\]

3. Add the equations to eliminate $y$:\\
\[
14x = 26 \rightarrow x = \frac{26}{14} = \frac{13}{7}
\]

4. Substitute $x = \frac{13}{7}$ into the first equation:\\
\[
2(\frac{13}{7}) + 3y = 8 \rightarrow \frac{26}{7} + 3y = 8
\]

5. Solve for $y$:\\
\[
3y = 8 - \frac{26}{7} = \frac{30}{7} \rightarrow y = \frac{10}{7}
\]

\textbf{Solution}: The solution to the system is:
\[
x = \frac{13}{7}, y = \frac{10}{7}
\]
\subsection{First Degree Inequalities}
A first degree inequality is an algebraic inequality that can be expressed in one of the forms:
\[
ax + b > 0, \quad ax + b < 0, \quad ax + b \geq 0, \quad \text{or} \quad ax + b \leq 0,
\]
where:
\begin{itemize}
    \item $a$ and $b$ are real numbers ($a \neq 0$)
    \item $x$ is the unknown variable
    \item The symbols $>$, $<$, $\geq$, $\leq$ represent inequality relations
\end{itemize}

\textbf{Solution Method}\\
Steps:\\
1. Isolate the term containing $x$ on one side of the inequality.\\

2. Divide or multiply both sides by the coefficient of $x$.\\
   \textbf{Important}: If multiplying or dividing by a negative number, reverse the inequality sign.\\

\textbf{Example of Application}

Consider the inequality:
\[
3x - 5 \leq 7
\]

1. Isolate $3x$:
\[
3x \leq 7 + 5 \quad \Rightarrow \quad 3x \leq 12
\]

2. Divide by 3 (since $3 > 0$, the inequality sign remains the same):
\[
x \leq \frac{12}{3} \quad \Rightarrow \quad x \leq 4
\]

\textbf{Solution}: All real numbers less than or equal to 4.\\
In interval notation: $(-\infty, 4]$.

\textbf{Special Cases}

1. \textbf{Multiplying/Dividing by Negative Numbers}:\\
For $-2x + 3 > 7$:
\[
-2x > 4 \quad \Rightarrow \quad x < -2 \quad \text{(sign reversed)}
\]

2. \textbf{No Solution Case}:\\
For $5x - 4 \geq 5x + 1$:
\[
-4 \geq 1 \quad \text{(false statement)}
\]

3. \textbf{Infinite Solutions Case}:\\
For $3x + 2 \leq 3x + 2$:
\[
0 \leq 0 \quad \text{(always true)}
\]
\subsection{Factoring (Trinomials, Special Products)}

\textbf{1. Factoring Trinomials}\\  
\textbf{A. Perfect Square Trinomial form}\\  
Form: $a^2 \pm 2ab + b^2 = (a \pm b)^2$\\  
Example:\\  
$x^2 + 6x + 9 = (x + 3)^2$\\  

\noindent\textbf{B. $x^2 + bx + c$ form}\\  
Find numbers $m$, $n$ where:  
$m + n = b$ and $m \cdot n = c$  
Example:  
$x^2 + 5x + 6 = (x + 2)(x + 3)$\\  

\noindent\textbf{C. $ax^2 + bx + c$ form}\\  
Use trial-and-error:  
$6x^2 + 7x - 3 = (3x - 1)(2x + 3)$  

---

\textbf{2. Special Products}  
\begin{tabular}{lll}
\textbf{Type} & \textbf{Formula} & \textbf{Example} \\
Square of Binomial & $(a \pm b)^2 = a^2 \pm 2ab + b^2$ & $(2x+3)^2 = 4x^2+12x+9$ \\
Difference of Squares & $a^2-b^2=(a+b)(a-b)$ & $x^2-16=(x+4)(x-4)$ \\
Sum/Diff of Cubes & $a^3 \pm b^3 = (a \pm b)(a^2 \mp ab + b^2)$ & $8x^3-27=(2x-3)(4x^2+6x+9)$ \\
\end{tabular}

---

\textbf{3. Solved Exercises}  
1. Factor $x^2 - 8x + 16$:  
\boxed{(x - 4)^2}  

2. Factor $2x^2 - 7x + 3$:  
\boxed{(2x - 1)(x - 3)}  

3. Expand $(3y + 4)^2$:  
\boxed{9y^2 + 24y + 16}

\subsection{Algebraic Identities}  
Essential formulas that relate algebraic expressions and simplify calculations.

---

\textbf{1. Square of a Binomial}  
\[
(a \pm b)^2 = a^2 \pm 2ab + b^2
\]  
\textbf{Rule:}\\  
- Square of first term\\  
- Twice the product of terms\\  
- Square of second term\\

\textbf{Example:}  
\[
(2x + 3)^2 = 4x^2 + 12x + 9
\]  

---

\textbf{2. Difference of Squares}  
\[
a^2 - b^2 = (a + b)(a - b)
\]  
\textbf{Rule:}\\
Factorizes into "sum times difference"\\  

\textbf{Example:}  
\[
9x^2 - 16 = (3x + 4)(3x - 4)
\]  

\textbf{3. Cube of a Binomial}  
\[
(a \pm b)^3 = a^3 \pm 3a^2b + 3ab^2 \pm b^3
\]  
\textbf{Rule:}\\
Cube of first term\\  
- 3 × (first term)² × (second term)\\  
- 3 × (first term) × (second term)²\\  
- Cube of last term  

\textbf{Example:}  
\[
(x + 2)^3 = x^3 + 6x^2 + 12x + 8
\]  

\textbf{4. Sum/Difference of Cubes}  
\[
a^3 \pm b^3 = (a \pm b)(a^2 \mp ab + b^2)
\]  
\textbf{Rule:}\\
"Binomial × Trinomial" factorization\\  
\textbf{Example (Sum):}  
\[
8 + y^3 = (2 + y)(4 - 2y + y^2)
\]  

\textbf{Example (Difference):}  
\[
27x^3 - 8 = (3x - 2)(9x^2 + 6x + 4)
\]  
\textbf{5. Product of Binomials with Common Term}  
\[
(x + a)(x + b) = x^2 + (a+b)x + ab
\]  
\textbf{Rule:}\\  
- Square of common term  
- Sum of non-common terms × common term  
- Product of non-common terms  

\textbf{Example:}  
\[
(x + 5)(x - 3) = x^2 + 2x - 15
\]

\section{Level 4 - Algebra II}
\subsection{Polynomials and operations with polynomials}
A polynomial expression has one or more terms with a coefficient, a variable base, and an exponent.
\begin{figure}[H]
    \centering
    \includegraphics[width=0.5\textwidth]{monomial.png}
    \caption{Quick overview of how the monomials looks like.}
    \label{fig:etiqueta}
\end{figure}
\noindent$3x^{4}$ is a monomial. This means that it only have one term. We also have binomials and trinomials.\\
$3x^{4}+2x$ is a binomial. The exponent of $2x$ is $1$.\\
$3x^{4}+2x+7$ is a trinomial. $7$ is a constant term. Also we can think of $7$ as an exponential term with an exponent of $0$.\\
\\
While we can add and subtract any polynomials, we can only combine like terms, which must have:
\begin{itemize}
    \item The same variable base
    \item The same exponent
\end{itemize}

\subsection{Combining Like Terms}

For example, we can combine the terms $2x^3$ and $ 4x^3 $ because they have the same variable base, $x$, and the same exponent, 3. However, we cannot combine the terms $2x^2$ and $2x^3$ because they have different exponents, 2 and 3.

\noindent When we combine like terms, only the coefficients change. Both the base and the exponent remain the same. For example, when adding $ 2x^3 $ and $ 4x^3 $, the $x^3 $ part of the terms remain the same, and we add only 2 and 4 when combining the terms:

\[
2x^3 + 4x^3 = (2 + 4)x^3 = 6x^3
\]

\noindent For example, when subtracting the polynomial \(-2x^2 - 7\), the negative sign from the subtraction is distributed to both \(-2x^2\) and \(-7\), which means:

\[
5x^2 - (-2x^2 - 7)
\]

\[
5x^2 + 2x^2 + 7
\]

\[
7x^2 + 7
\]

\noindent To multiply polynomials, we use the distributive property (also called the FOIL method for binomials). Here's how it works:

\[
(2x + 3)(x - 4)
\]

Multiply each term in the first polynomial by each term in the second:

\[
= 2x \cdot x + 2x \cdot (-4) + 3 \cdot x + 3 \cdot (-4)
\]

\[
= 2x^2 - 8x + 3x - 12
\]

Combine like terms:

\[
= 2x^2 - 5x - 12
\]

\boxed{\textbf{Key Steps:}}
1. Multiply each term in the first polynomial by each term in the second
2. Add all the resulting terms together
3. Combine like terms to simplify

\textbf{Remember:} When multiplying variables with exponents, add the exponents:
\[
x^a \cdot x^b = x^{a+b}
\]
\end{document}