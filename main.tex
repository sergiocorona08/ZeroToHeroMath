\documentclass{article} % Tipo de documento
\usepackage[a4paper, margin=1.5cm]{geometry} % Márgenes de 1.5 cm
\usepackage[utf8]{inputenc} % Codificación
\usepackage[spanish]{babel} % Idioma
\usepackage[T1]{fontenc} %Nos sirve para agregar texto
\usepackage{amsmath, amssymb}


\title{Zero to Hero, Math}
\author{Sergio Augusto Macías Corona}
\date{\today} % Fecha actual

\begin{document}

\maketitle

\section{Introduction}

The following PDF will be a very useful tool created by me to help me along my professional carreer. It will also help me review anything I might have forgotten over time abouth Math, and why not? Also help me brush up on my English writing. If you are reading this, maybe it´s just me, but if not. Just want to tell you that I´m shit at Math but I hope that with hard work, sweat, and many tears, me, but also you can learn everything you don't understand about math and be able to complete your college, high school, or whatever else is in a more enjoyable way, and not suffer as much as I did. With love, sergio\_corona88 at Instagram.

\section{Level 0: Absolute Fundamentals (Pre - Math)}
\subsection{Natural numbers}
Natural numbers are the set of all the numbers from one to infinity excluding zero. (1 - $\infty$).The set of natural numbers is denoted by the symbol, N. So, the set of natural numbers is represented as $N = [1,2,3,4,5 \dots \infty$]. 

\subsection{Addition and subtraction}
The two simplest operations: addition is used to find the total between two numbers. $a = b + c$ or $4 + 2 = 6$ Subtraction, on the other hand, is used to find the difference between two numbers. $a = b - c$ or $6 - 4 = 2$

\subsection{Multiplication and division}
Multiplication and division are directly related because they are essentially inverse operations. Multiplication seeks to join equal groups, while division seeks to separate these equal groups.

Let's start with the most basic. If we have $4 \times 5 = 20$, their inverse relationships (in the form of division) would be the following:

\[
20 \div 5 = 4
\]

\[
20 \div 4 = 5
\]

De igual forma, si tomamos la división $30 \div 3 = 10$, sus relaciones inversas (en forma de multiplicación) serían las siguientes:
\[
3 \times 10 = 30
\]

\[
10 \times 3 = 30
\]
In both examples, we can see that we use the same three numbers. This is because when we multiply two numbers (which we call factors), we obtain a result that we call the product. If we divide that number by one of the factors, we obtain the other factor as the result.

\subsection{Even and odd numbers}
Basically, an even number is one that can be divisible by 2 and generates a remainder of 0. An even number can be easily divided into equal groups. $(2,4,6,8,10\dots)$.
In the other hand, odd numbers are numbers that are not evenly divisible by 2. When divided by 2, even numbers leave a remainder of 1. $(1,3,4,5,7,9\dots)$.

\subsection{Number comparison}
Comparing numbers in the process in which we can determine whether a number is smaller, greater, or equal to another number according to their values. The symbols used for comparing numbers are “>”, which means “greater than”; “<”, which means “less than”; and “=”, which means “equal to". We also have a subcategory type that allows us to determine whether a number is, for example, greater than another but also equal. This is defined by the following symbol "$\geq$" which means "greater than or equal to". And finally, "less than or equal to," which is defined by the following symbol "$\leq$".

\[
6 > 3
\]
6 is greater than 3 wich is true. 

\[
2 > 9
\]
2 is smaller than 9 wich is true.

\[
5 = 5
\]
5 is equal to 5 wich is also true.

\subsection{Zero and negative numbers}
Positive numbers represent an amount of something. Negative numbers represent taking away an amount of something. $(-1,-2,-3,-4,-5\dots)$ For example if we acquire some amount of money we can represent it by a positive number. If we then spend some of that money we can represent it by a negative number.

Zero as a number represents a starting point when we have neither acquired something nor had any of it taken away. Is a digit that is crucial to place value number systems. It allows us to represent numbers that have none of a particular place value.

\section{Level 1: Basic arithmetic}

\subsection{Whole numbers}
The whole numbers are the part of the number system which includes all the positive integers from 0 to infinity. These numbers exist in the number line. Hence, they are all real numbers. $(0, 1, 2, 3, 4 \dots \infty)$

\subsection{Fractional numbers}
The fractional numbers are numbers that represent one or more parts of a unit that has been divided in equal parts. They are figured out by two whole numbers (the fraction terms) that are separated by a horizontal line (the fraction line). The number above the line (the numerator) can be every whole number and the number below the line (the denominator) should be different from zero.

There can be different types of fractional numbers which are named differently as follows.\\

\textbf{Proper Fraction:} the number is inferior to the denominator, for instance:
\[
\frac{3}{4}
\]

\textbf{Improper Fraction:} the numerator is superior to the denominator, for instance:
\[
\frac{9}{2}
\]

\textbf{Mixed Fraction or Mixed Numeral:} it is composed of a whole part and a fractional one, for instance:
\[
2\frac{1}{3}
\]

\textbf{Equivalent Fractions:} fractions that keep on the same proportion of another fraction, for instance:
\[
\frac{5}{2} = \frac{10}{4}
\]

\textbf{Irreducible Fraction:} it cannot be simplified, for instance:
\[
\frac{4}{3}
\]

\textbf{Decimal Fraction:} the denominator is a power whose base is $10(10, 100, 1000 \dots)$, for instance:
\[
\frac{8}{10}
\]
Not every number written as a fraction is a fractional number,the reason is that fractional numbers represent one or more parts of a whole. For example, the fraction $\frac{10}{2}$ is written as a fraction, but it is not a fractional number because it simplifies to $5$, which is a whole number and does not represent a part of a whole. Similarly, $\frac{\sqrt{2}}{3}$ is expressed as a fraction, but it is not a fractional number because the numerator ($\sqrt{2}$) is not an whole number.

\subsection{Decimal numbers}
Decimals are one of the types of numbers, which has a whole number and the fractional part separated by a decimal point. The dot present between the whole number and fractions part is called the decimal point. For example, $34.5$ is a decimal number.

There can be different types of decimal numbers which are named differently as follows.\\

\textbf{Recurring Decimal Numbers:} (Repeating or Non-Terminating Decimals)

\[
3.125125 (Finite)
\]
\[
3.121212121212\dots (Infinite)
\]

\textbf{Non-Recurring Decimal Numbers:} (Non Repeating or Terminating Decimals)

\[
3.2376 (Finite)
\]
\[
3.137654\dots (Infinite)
\]

\textbf{Decimal Fraction:} (It represents the fraction whose denominator in powers of ten.)

\[
81.75 = \frac{32425}{1000} 
\]

\subsection{Percentages}
A percentage is a number, or ratio, that expresses a fraction out of 100. Percentages are easy to recognize because they are always followed by a percent sign (\%). A percentage is an alternative way to represent a fraction out of 100 instead of using a traditional fraction format. For example, we can write $\frac{1}{2}$ or $\frac{50}{100}$ to represent a half fraction, using percentage we can also write 50\%.

\subsection{Proportions and ratios}
A ratio compares two or more quantities, showing their relative sizes. It can be written in different ways:

\textbf{Colon notation:} $a : b$ (example, $2:3$)

\textbf{Fraction form:} $a / b$ (example, $2 / 3$)

\textbf{'to' phrasing:} $a / b$ (example, 2 to 3)\\

A proportion is a statement that two ratios (or fractions) are equal. It tells us that two different comparisons have the same relationship.
If $\frac{a}{b}$ and $\frac{c}{d}$ represent the same relative amount, then:

\[
\frac{a}{b} = \frac{c}{d}
\]

\subsection{Least common multiple (LCM) and greatest common divisor (GCD)}
The GCD of two or more integers is the largest positive integer that divides each of the given numbers without leaving a remainder. In other words, it is the greatest common factor shared by the numbers.

\textbf{For example to:}\\
\noindent The GCD of:
\[
\{34, 56\} \text{ is } 22
\]
\noindent The GCD of:
\[
\{64, 96\} \text{ is } 32
\]

The LCM of two or more integers is the smallest positive integer that is divisible by each of the given numbers. It is the least common multiple shared by the numbers.

\textbf{For example to:}\\
\noindent The LCM of:
\[
\{34, 56\} \text{ is } 952
\]
\noindent The LCM of:
\[
\{64, 96\} \text{ is } 192
\]


\subsection{Order of operations (PEMDAS)}
The order of operations is a set of rules that tells us the correct sequence to solve math problems with multiple operations (like addition, subtraction, multiplication, division, exponents, etc.). Without it, the same problem could have different and wrong answers.
So here I show a quick guide of what the order of operations is:

\textbf{Solve parentheses first:}\\
\[
4 \times (5 + 3) = 4 \times (8) = 32
\]

\textbf{Exponents (Powers, Roots) before Multiply, Divide, Add or Subtract:}\\
\[
5 \times 2^{2} = 5 \times 4 = 20
\]

\textbf{Multiply or Divide before you Add or Subtract:}\\
\[
2 + 5 \times 3 = 2 \times 15 = 17 
\]

\textbf{Otherwise just go left to right:}\\
\[
30 \div 5 = 6 \times 3 = 18
\]

\subsection{Rounding and estimating}
\textbf{Rounding numbers}\\
We do not always need to give exact answers to problems. Rounding means making a number simpler (less precise) while keeping it close to the original value. We often round to a specific place value.

\textbf{Rules for Rounding:}\\
\textbf{1.-} Identify the place value you're rounding to (tens, hundreds, decimals).\\
\textbf{2.-} Look at the digit to the right of that place:\\
If it's 5 or higher, round up the target digit by 1.\\
If it's 4 or lower, Keep the target digit the same.
\textbf{3.-} Replace all digits to the right with zeros (for whole numbers) or drop them (for decimals).\\
\\
\textbf{Estimating}\\
Estimating involves using rounded numbers to quickly approximate calculations. It´s useful for checking if an exact answer is reasonable.

\textbf{Estimating methods:}\\
\textbf{Front-End Estimation:} Use the first digits and adjust.\\
\[
327 + 589 \approx 300 + 500 = 800
\]

\noindent \textbf{Rounding First:} Round all numbers before calculating.\\
\[
23 \times 48 \approx 20 \times 50 = 1000
\]

\section{Level 2: Advanced Arithmetic / Pre - Algebra}

\end{document}
