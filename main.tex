\documentclass{article} % Tipo de documento
\usepackage[a4paper, margin=1.5cm]{geometry} % Márgenes de 1.5 cm
\usepackage[utf8]{inputenc} % Codificación
\usepackage[spanish]{babel} % Idioma
\usepackage[T1]{fontenc} %Nos sirve para agregar texto
\usepackage{amsmath, amssymb}


\title{Zero to Hero, Math}
\author{Sergio Augusto Macías Corona}
\date{\today} % Fecha actual

\begin{document}

\maketitle

\section{Introduction}

The following PDF will be a very useful tool created by me to help me along my professional carreer. It will also help me review anything I might have forgotten over time abouth Math, and why not? Also help me brush up on my English writing. If you are reading this, maybe it´s just me, but if not. Just want to tell you that I´m shit at Math but I hope that with hard work, sweat, and many tears, me, but also you can learn everything you don't understand about math and be able to complete your college, high school, or whatever else is in a more enjoyable way, and not suffer as much as I did. With love, sergio\_corona88 at Instagram.

\section{Level 0: Absolute Fundamentals (Pre - Math)}
\subsection{Natural numbers}
Natural numbers are the set of all the numbers from one to infinity excluding zero. (1 - $\infty$).The set of natural numbers is denoted by the symbol, N. So, the set of natural numbers is represented as $N = [1,2,3,4,5$\dots $ \infty$]. 

\subsection{Addition and subtraction}
The two simplest operations: addition is used to find the total between two numbers. $a = b + c$ or $4 + 2 = 6$ Subtraction, on the other hand, is used to find the difference between two numbers. $a = b - c$ or $6 - 4 = 2$

\subsection{Multiplication and division}
Multiplication and division are directly related because they are essentially inverse operations. Multiplication seeks to join equal groups, while division seeks to separate these equal groups.

Let's start with the most basic. If we have $4 \times 5 = 20$, their inverse relationships (in the form of division) would be the following:

\[
20 \div 5 = 4
\]

\[
20 \div 4 = 5
\]

De igual forma, si tomamos la división $30 \div 3 = 10$, sus relaciones inversas (en forma de multiplicación) serían las siguientes:
\[
3 \times 10 = 30
\]

\[
10 \times 3 = 30
\]
In both examples, we can see that we use the same three numbers. This is because when we multiply two numbers (which we call factors), we obtain a result that we call the product. If we divide that number by one of the factors, we obtain the other factor as the result.

\subsection{Even and odd numbers}
Basically, an even number is one that can be divisible by 2 and generates a remainder of 0. An even number can be easily divided into equal groups. $(2,4,6,8,10\dots)$.
In the other hand, odd numbers are numbers that are not evenly divisible by 2. When divided by 2, even numbers leave a remainder of 1. $(1,3,4,5,7,9\dots)$.

\subsection{Number comparison}
Comparing numbers in the process in which we can determine whether a number is smaller, greater, or equal to another number according to their values. The symbols used for comparing numbers are “>”, which means “greater than”; “<”, which means “less than”; and “=”, which means “equal to". We also have a subcategory type that allows us to determine whether a number is, for example, greater than another but also equal. This is defined by the following symbol "$\geq$" which means "greater than or equal to". And finally, "less than or equal to," which is defined by the following symbol "$\leq$".

\[
6 > 3
\]
6 is greater than 3 wich is true. 

\[
2 > 9
\]
2 is smaller than 9 wich is true.

\[
5 = 5
\]
5 is equal to 5 wich is also true.

\subsection{Zero and negative numbers}
Positive numbers represent an amount of something. Negative numbers represent taking away an amount of something. $(-1,-2,-3,-4,-5\dots)$ For example if we acquire some amount of money we can represent it by a positive number. If we then spend some of that money we can represent it by a negative number.

Zero as a number represents a starting point when we have neither acquired something nor had any of it taken away. Is a digit that is crucial to place value number systems. It allows us to represent numbers that have none of a particular place value.

\section{Level 1: Basic arithmetic}

\end{document}
